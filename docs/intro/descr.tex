\documentclass[../relatorio.tex]{subfiles}
\begin{document}
O enunciado escolhido, \textbf{Linguagem de templates} 
(inspirada nos \textit{templates Pandoc}),
propõe a criação de um compilador de \textit{templates}.
Deste modo, utiliza-se:
\begin{enumerate}
    \item um \textit{Template}.
    \item um Dicionário.
\end{enumerate}
\dots para construir o texto final, um \textbf{ficheiro \textit{markup}}, 
com todos os dados submetidos pelo utilizador.  

Os \textit{templates} a usar
devem incluir um conjunto de regras, para permitir 
substituir os dados submetidos pelo utilizador 
nos parâmetros desejados. 
Neste sentido, propõe-se o desenvolvimento da \textbf{gramática}
que cada \textit{template} deve seguir, 
independemente da linguagem que está a utilizar.

Para a criação do \textbf{compilador}, segue-se a criação do:
\begin{itemize}
    \item Analisador Léxico. 
    \item \textit{Parser}.
\end{itemize}
\dots que permite capturar os campos de 
\textbf{metadados} dos \textit{templates}, sem afetar
o restante texto, para posteriormente os poder tratar,
gerando o texto final. 
Para tal deve-se recorrer ao módulo \textit{PLY} do \textit{Python},
nomeadamente o gerador de analisador léxicos - \textit{Lex} - e
o gerador de compiladores (com base em gramáticas tradutoras) - \textit{Yacc}.

O \textbf{dicionário} corresponde aos dados 
a serem inseridos no \textit{template}. 
No projeto a realizar utilizar-se-á ficheiros 
\textit{YAML}, usufruindo da organização 
estrutural da linguagem.

\end{document}