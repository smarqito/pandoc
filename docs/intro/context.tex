\documentclass[../relatorio.tex]{subfiles}
\begin{document}
O presente relatório foi escrito no âmbito da Unidade Curricular (UC) de Processamento de Linguagens (PL), apresentando 
como objetivo a descrição da solução desenvolvida para o segundo trabalho prático, assim como as decisões 
tomadas para a sua conceção.

A solução é relativa ao primeiro enunciado: \textbf{Linguagens de Templates} 
(inspirada nos \textit{templates Pandoc}).

Neste sentido, é proposto o desenvolvimento de um compilador de \textit{templates},
de modo a criar ficheiros \textit{markup}, a partir do processamento de
\textit{templates} pré-definidos. 
Para tal usou-se a linguagem \textit{YAML} para guardar toda a 
informação a ser usada no preenchimento do \textit{template}.

Assim, engloba o desenvolvimento do \textbf{analisador léxico} e o \textbf{parser}, 
bem como a gramática e a estrutura da árvore sintática criadas  
para poder gerar o texto final. 

\end{document}