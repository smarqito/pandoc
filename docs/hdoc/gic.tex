\documentclass[../relatorio.tex]{subfiles}
\begin{document}

Com o intuito de descrever a estrutura hierárquica da linguagem
definida para os \textit{templates}, construiu-se a \textbf{gramática} 
a usar no compilador. 

Como o programa na íntegra dependia das decisões tomadas durante
a definição da gramática, foi o primeiro passo no processo de desenvolvimento
da solução. 

A \textbf{gramática independente de contexto} (GIC) usada 
no âmbito do presente projeto encontra-se em \ref{fig:grammar}.

Mediante os símbolos \textbf{terminais}, \textbf{literais} e as palavras 
\textbf{reservadas} definidas no contexto do analisador léxico do 
compilador, em \ref{subsec:lex}, seguem-se as \textbf{produções}:

\subsubsection{\textit{Document}}\label{grm:doc}
Correspondente à primeira produção criada no contexto gramatical,
\textit{Document} caracteriza o \textit{template} na sua totalidade,
com todos os elementos constituintes (\ref{grm:elems}).
A produção dada à sua simplicidade permaneceu imutável desde o ínicio
do desenvolvimento do projeto até ao fim, sendo todas as alterações 
feitas ao nível atómico das produções.

\subsubsection{\textit{Elements}} \label{grm:elems}
Os elementos que compõem um documento são caracterizados como uma lista 
de elemento (\ref{grm:elem}). 
Não é do interesse do projeto tratar \textit{templates} vazios, pelo que 
é obrigatório um documento ter \textbf{pelo menos} um 
elemento constituinte, seja qual for. 

Como o \textit{parser} usado (\textit{Yacc}) tem um reconhecimento \textbf{LALR}
é vantajoso o uso de \textbf{recursividade à esquerda} para a definição da lista.
Deste modo, será mais eficiente o \textit{shift-reduce parsing}, para além
de permitir gramáticas com produções a começar no mesmo símbolo.

\subsubsection{\textit{Element}}\label{grm:elem}
Componenentes primitivos do documento. 
Um elemento pode ser do tipo:
\begin{itemize}
    \item \textit{Statement}    (Stmt)
    \item Texto                 (TEXT)
    \item Variável              (Var)
    \item \textit{Backslash}    (BACK)
    \item Iterável              (It)
    \item \textit{Nesting}      (Nesting)
\end{itemize}

O elemento \textbf{Texto} representa a parte inalterável do 
ficheiro. 
Os restantes são respetivos aos campos de \textbf{metadados}
do \textit{template}.
Devido à organização de um documento, um novo elemento integrante 
poderá ser facilmente inserido
na gramática da linguagem, acrescentando uma nova produção.

Com a exceção do \textit{TEXT} e \textit{BACK},
tratam-se todos de símbolos não terminais, sendo facilitada
o \textbf{decisão} perante a produção a escolher; se no começo 
de uma estiver a identificar corretamente os parâmetros, 
as restantes produções já não são válidas.

\subsubsection{\textit{Stmt}}\label{grm:stmt}
Um \textit{Statement} refere-se a 3 operações:
\begin{enumerate}
    \item[\textit{If}]         {Condicional}
    \item[\textit{For}]        {Ciclo}
    \item[\textit{Subtemplate}]{Ficheiros adicionais}
\end{enumerate}
\dots foi realizada uma fatorização à esquerda para cada 
\textit{statement} possível, passando a ter apenas 
\textbf{3 símbolos não terminais}.

\subsubsection{\textit{IfStmt}} \label{grm:ifstmt}
Um \textit{IfStmt} trata-se de uma transformação
condicional.

Apresenta-se como:
\begin{minted}{bash}    
$if(cond)$
    body
$endif$
\end{minted}
\dots sendo que o \textit{body} só vai ser executado
se $cond=True$; em que a variável \textit{cond} é do 
tipo \ref{grm:conds}.
Um $body$ pode ser constituído por 
vários elementos a serem executados (\ref{grm:elems}).
É permitido existir casos opcionais, 
no seguimento da condição falhar, 
$cond=False$, na forma de \textit{ElseIfs} ou \textit{Elses}.
A marca \textit{endif} simboliza o término da operação.

\subsubsection{\textit{Else}} \label{grm:else}
A produção \textit{Else} refere-se aos casos opcionais 
dentro de uma condicional, na ocorrência de falhar a 
condição de entrada.

Segue-se:
\begin{enumerate}
    \item \textit{Else :} No caso de tudo falhar, executa o \textit{body} dentro do campo
                          e acaba a operação.
    \item \textit{ElseIf :} Apresenta uma nova condição, se entrar, executa o seu \textit{body}
                            e acaba a operação.
    \item \textit{Empty :} Não existem casos opcionais.
\end{enumerate}

A estrutura de um \textit{ElseIf} e \textit{Else}:
\begin{minted}{bash}    
    $if(cond)$
        If_body
    $elseif(cond_elseif)$
        elseif_body
    $else$
        Else_body
    $endif$
\end{minted}

Convencionou-se que um \textit{Else} seria a última operação dentro de um \textit{If}, no 
caso de tudo falhar. 
Por este motivo, como podem existir $n$ \textit{ElseIfs}, foi necessária a utilização 
de \textbf{recursividade à direita}. 
Deste modo, é garantida uma lista de \textit{ElseIfs}, em que o \textit{Else}
é a última operação, no caso de existir.

Apesar de não ser tão eficiente, não existem problemas relacionados com a 
utilização de recursividade à direita num \textit{parser LALR}.

\subsubsection{\textit{ElseIf}} \label{grm:elseif}
Decidiu-se representar os \textit{ElseIfs} numa produção 
à parte, sendo um não terminal no contexto da produção 
\textit{Else}(\ref{grm:else}).
A sua estrutura assemelha-se à de um \textit{If}, com 
a exceção de utilizar uma nova palavra reservada e não 
terminar obrigatoriamente num \textit{Endif}.

\subsubsection{\textit{ForStmt}} \label{grm:forstmt}
O elemento \textit{ForStmt} representa uma transformação cíclica.
Para tal recebe uma condição, \ref{grm:conds}, a iterar 
e os elementos a executar durante o período de vida, \ref{grm:elems}.

A operação deve obrigatoriamente terminar com \textit{endfor}.

Deste modo, apresenta-se como:
\begin{minted}{bash}
    $for(cond)$
        for_body $SEP$ separador
    $endfor$
\end{minted}

Adicionalmente, pode-se colocar um \textbf{separador} a ser adicionado
em cada iteração à frente do corpo do ciclo, com a exceção da última.
Para tal foi criada a produção \textit{Sep}.
O separador é opcional, por isso tem um campo vazio.

\subsubsection{\textit{StmtSubtemplate}} \label{grm:sub}
Para permitir utilizar outros \textit{templates} e substituir
o seu texto no atual, adicionou-se a produção \textit{StmtSubtemplate}.

Utilizou-se os parênteses $()$ como sinal para a utilização de um 
\textit{subtemplate}.
Segue-se:
\begin{minted}{bash}
    $subtemplate()$
    $subtemplate()\pipe$
\end{minted}

Adicionalmente, estes podem sofrer alterações através de \textit{pipes} (\ref{grm:pipes}),
utilizando o não terminal, no final da produção.
\dots \textit{pipes} devem afetar todo o ficheiro, pelo que este deve ser 
percorrido pelo \textit{parser}, como o \textit{template} principal.

Uma distinção dos \textit{subtemplates}
teve de ser feita para permitir o uso a partir
de um iterável, \ref{grm:it},
nomeadamente no interior de ciclos \textit{for}, não sendo permitidos \textit{pipes}.
Assim, foi criada a produção \textit{SubIt}.

\subsubsection{\textit{It}} \label{grm:it}
Para permitir dentro das operações de \textbf{metadados} acessar à iteração
de um objeto, utiliza-se a diretiva \textit{it},
representada na produção \textit{It}.

Seria no sentido:
\begin{minted}{bash}
    $it$
    $it:subtemplate()$
    $it.var:subtemplate()$
    $it.var$
    $it.var[begin_range, end_range]$
    $it.[begin_range, end_range]$
\end{minted}

Neste caso, o \textit{it} pode iterar os objetos que 
movimentam-se dentro dos metadados.
São utilizados com grande efeito dentro dos ciclos \textit{for}, 
para permitir obter o próximo valor dentro de um 
iterável, em cada nova iteração do \textit{loop}.

É notável a fatorização à esquerda efetuada entre todas as possíveis 
opções dentro do \textit{it}.
Como dito anteriormente, o \textit{parser} beneficia deste 
comportamente, na decisão do caminho da árvore sintática.

\subsubsection{\textit{Nesting}} \label{grm:nest}
Para permitir conteúdo \textit{nested} dentro do documento, i.e
a preservação da identação das linhas subsequentes, 
foram criadas as produções para o \textit{nesting}.

Utiliza-se a diretiva - \^ - para separar o elemento para o qual todas as 
linhas se vão indexar, \textit{Var}, e o restante texto, em \textit{NestElems}.

Para tomar partido do reconhecedor utilizado pelo \textit{yacc}, 
definiu-se à esquerda a recursividade da lista de elementos, \textit{NestElems}.
Estes podem ser \textbf{texto} ou \textbf{variáveis}.

\dots apesar do \textit{NestElems} representar, na íntegra, a produção 
\textit{elems} (\ref{grm:elems}), foi necessário a criar para evitar problemas 
com \textbf{reduções}, sendo provocada uma circularidade não controlável.

\subsubsection{\textit{Conds}} \label{grm:conds}
As variáveis utilizadas na entrada das operações de \textit{If} e
\textit{For} são apresentadas com a produção \textit{cond}.

No contexto das operações \textit{for} é dada a opção 
de inserir um \textit{range} de valores; para permitir
mais flexibilidade ao utilizador.

\subsubsection{\textit{Vars}} \label{grm:vars}
O elemento atómico para representar variáveis no documento.

Uma variável é caracterizada pelo seu nome, um \textit{ID}.
Contudo, no caso de mapas, os valores devem ser acessados 
com $.key\_name$.
Neste sentido, estamos a manipular os valores cujo nome da 
chave no mapa é \textit{key\_name}.

Adicionalmente, como é possível haver aninhamento de mapas, 
recorreu-se à recursividade à esquerda.
Aproveitando o facto do \textit{parser} ser \textit{Bottom-Up},
as variáveis podem acessar os valores de qualquer mapa 
existente no dicionário.

Mais tarde, devido à implementação de \textit{pipes}(\ref{grm:pipes}), 
as produções foram alteradas para permitir uma variável 
ser modificada; foi criada \textit{VarAtomic} para representar
apenas as variáveis.
 
\subsubsection{\textit{Pipes}} \label{grm:pipes}
Foram implementados um conjunto de \textit{pipes}, 
como apresentados em \ref{subsec:pipes}.

É permitido \textbf{aninhamento de pipes}, sendo a produção
\textit{pipes} uma lista.
Por esse motivo existe um caso \textbf{vazio}.

Devido ao reconhecedor utilizado pelo \textit{Yacc} ser \textit{Bottom-Up}
é preferível utilizar \textbf{recursividade à esquerda}, sendo por isso 
aplicada em todas as produções.

Neste caso, podem ser apresentados como:
\begin{minted}{bash}
    \uppercase
    \uppercase\center 20 "|" "|"
\end{minted}
\dots adicionalmente apresenta-se \textit{pipes} com argumentos, sendo 
dinâmicos no seu \textit{output}.

A adição dos \textit{pipes} nas operações superiores provocou 
alguns problemas do âmbito \textit{reduce-reduce} e \textit{shift-reduce},
sendo rapidamente tratados; nomeadamente na produção \textit{StmtSubtemplate} (\ref{grm:sub}).

\subsubsection{\textit{Num}} \label{grm:num}
Dado a ser possibilitada a utilização de \textit{range} nos iteráveis (\ref{grm:it}),
foi criada uma produção para representar os parâmetros recebidos, do tipo numérico.

A necessidade advém de ser oferecida a opção de não inserir um valor, para representar
os extremos de um objeto a iterar.
Por esse motivo, \textit{Num} têm uma produção a vazio.

\subsubsection{\textit{Newline}} \label{grm:nl}
A produção é utilizada para o tratamento dos caracteres \textit{newline}
em todo o ficheiro.
Ao nível da análise léxica são contabilizados os caracteres \textit{newline}
lidos no ficheiro original, sendo guardado o seu número.
Posteriormente, serão adicionados no texto final, nas 
posições apropriadas para re-estabelecer o formato do \textit{template}.

É possível não existir nenhum caracter \textit{newline} no ficheiro original,
no contexto de uma produção da gramática, pelo que existe o caso \textbf{vazio}.

\end{document}