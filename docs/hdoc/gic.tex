\documentclass[../relatorio.tex]{subfiles}
\begin{document}

Com o intuito de descrever a estrutura hierárquica da linguagem
definida para os \textit{templates}, construiu-se a \textbf{gramática} 
a usar no compilador. 

Como o programa na íntegra dependia das decisões tomadas durante
a definção da gramática, foi o primeiro passo no processo de desenvolvimento
da solução. 

Usufruindo da biblioteca \textit{PLY} para facilmente aplicar as 
regras sintáticas, a \textbf{gramática independente de contexto} (GIC) usada 
no âmbito do presente projeto encontra-se em \ref{fig:grammar}.

Mediante os símbolos \textbf{terminais}, \textbf{literais} e as palavras 
\textbf{reservadas} definidas no contexto do analisador léxico do 
compilador, em \ref{subsec:lex}, seguem-se as \textbf{produções}:

\subsubsection{\textit{Document}}
Correspondente à primeira produção criada no contexto gramatical,
\textit{Document} caracteriza o \textit{template} na sua totalidade,
com todos os elementos constituintes.
A produção dada à sua simplicidade permaneceu imutável desde o ínicio
do desenvolvimento do projeto até ao fim, sendo todas as alterações 
ao nível atómico das produções.

\subsubsection{\textit{Elements}}
Os elementos que compõem um documento são caracterizados como uma lista 
de elemento. 
Não é do interesse do projeto tratar \textit{templates} vazios, pelo que 
é obrigatório um documento ter \textbf{pelo menos} um 
elemento constituinte, seja qual for. 

\subsubsection{Elemento}



\end{document}