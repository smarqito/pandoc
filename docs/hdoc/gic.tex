\documentclass[../relatorio.tex]{subfiles}
\begin{document}

Com o intuito de descrever a estrutura hierárquica da linguagem
definida para os \textit{templates}, construiu-se a \textbf{gramática} 
a usar no compilador. 

Como o programa, na íntegra, dependia das decisões tomadas durante
a definição da gramática, foi o primeiro passo no processo de desenvolvimento
da solução.

A \textbf{gramática independente de contexto} (GIC) usada 
no âmbito do presente projeto encontra-se em \ref{fig:grammar}.

Mediante os símbolos \textbf{terminais}, \textbf{literais} e as palavras 
\textbf{reservadas} definidas no contexto do analisador léxico do 
compilador, em \ref{subsec:lex}, seguem-se os \textbf{não terminais}:

\subsubsection{\textit{Document}}\label{grm:doc}
Correspondente ao \textbf{axioma},
\textit{Document} caracteriza o \textit{template} na sua totalidade,
com todos os elementos constituintes (\ref{grm:elems}).
A produção, dada à sua simplicidade, permaneceu imutável durante todo o período de
desenvolvimento do projeto, sendo todas as alterações 
feitas ao nível atómico das produções.

\begin{minted}{bash}
Document -> Elems
\end{minted}

\subsubsection{\textit{Elements}} \label{grm:elems}
Os elementos que compõem um documento são caracterizados como uma lista 
de \textbf{Element} (\ref{grm:elem}). 
Não é do interesse do projeto tratar \textit{templates} vazios, pelo que 
é obrigatório um documento ter \textbf{pelo menos} um 
elemento constituinte, seja qual for. 

Como o \textit{parser} usado (\textit{Yacc}) tem um reconhecimento \textbf{LALR}
é vantajoso o uso de \textbf{recursividade à esquerda} para a definição da lista.
Deste modo, será mais eficiente o \textit{shift-reduce parsing}, para além
de permitir gramáticas com produções a começar no mesmo símbolo.

\begin{minted}{bash}
Elems   ->  Elems Elem
        |   Elem
\end{minted}

\subsubsection{\textit{Element}}\label{grm:elem}
Componenentes primitivos do documento. 
Um elemento pode ser do tipo:
\begin{itemize}
    \item \textit{Statement}    (Stmt)
    \item Texto                 (TEXT)
    \item Variável              (Var)
    \item \textit{Backslash}    (BACK)
    \item Iterável              (It)
    \item \textit{Nesting}      (Nesting)
    \item Comentário            (COMMENT)
\end{itemize}

O elemento \textbf{Texto} representa a parte do 
ficheiro que não contém variáveis dinâmicas.
Os restantes são respetivos aos campos de \textbf{metadados} do \textit{template}.
Devido à organização de um documento, um novo elemento integrante 
poderá ser facilmente inserido na gramática da linguagem, acrescentando uma nova produção.

\dots um exemplo concreto desta organização é relativo ao 
elemento \textit{Comment}, que foi adicionado 
posteriormente, no final da projeto.

\begin{minted}{bash}
Elem    ->  Stmt Newline
        |   TEXT NewLine
        |   Var  NewLine
        |   BACK NewLine
        |   It
        |   Nesting
        |   OCOMMENT COMMENT CCOMMENT Newline
\end{minted}

\subsubsection{\textit{Stmt}}\label{grm:stmt}
Um \textit{Statement} refere-se a 3 operações:
\begin{enumerate}
    \item[\textit{If}]         {Condicional}
    \item[\textit{For}]        {Ciclo}
    \item[\textit{Subtemplate}]{Ficheiros adicionais}
\end{enumerate}
\dots foi realizada uma fatorização à esquerda para cada 
\textit{statement} possível, passando a ter apenas 
\textbf{3 símbolos não terminais}.

\subsubsection*{\textit{If Stmt}} \label{grm:ifstmt}
\begin{minted}{bash}
    If      ->  IF OPAR Cond CPAR Newline Elems Else ENDIF
\end{minted}
Um \textit{If Stmt} trata-se de uma transformação
condicional, em que o corpo (\textit{body}) apenas será executado
se $cond\ is\ True$.
\begin{minted}{bash}    
$if(cond)$
    body
$endif$
\end{minted}
O corpo pode ser constituído por vários elementos a serem executados.
É permitido existir casos opcionais, no seguimento da condição falhar, 
$cond=False$, na forma de \textit{ElseIf} ou \textit{Else}.
A marca \textit{endif} simboliza o término da operação.

\subsubsection{\textit{Else} e \textit{ElseIf}} \label{grm:else}
A produção \textit{Else} refere-se aos casos opcionais 
dentro de uma condicional, na ocorrência de falhar as
condições anteriores.

Segue-se:
\begin{enumerate}
    \item \textit{ElseIf :} Apresenta uma nova condição, se entrar, executa o seu \textit{body}
                            e acaba a operação.
    \item \textit{Else :} No caso das condições anteriores falharem, executa o \textit{body} dentro do campo
                          e acaba a operação.
    \item \textit{Empty :} Não existem casos opcionais.
\end{enumerate}

A estrutura de um \textit{ElseIf} e \textit{Else}:
\begin{minted}{bash}    
    $if(cond)$
        If_body
    $elseif(cond_elseif)$
        elseif_body
    $else$
        Else_body
    $endif$
\end{minted}

Convencionou-se que um \textit{Else} seria a última operação dentro de um \textit{If}, no 
caso de tudo falhar. 
Por este motivo, como podem existir $n$ \textit{ElseIf}, optou-se pela utilização 
de \textbf{recursividade à direita} para manter a gramática mais próxima da sintaxe concreta. 
Deste modo, é garantida uma lista de \textit{ElseIf}, em que o \textit{Else}
é a última operação, no caso de existir.

\subsubsection{\textit{ForStmt}} \label{grm:forstmt}
\begin{minted}{bash}
For     -> FOR OPAR Cond CPAR Newline Elems Sep ENDFOR  
\end{minted}

O elemento \textit{ForStmt} representa uma transformação cíclica.
Para tal recebe uma condição, \ref{grm:conds}, a iterar 
e os elementos a compilar em cada iteração.

A operação deve obrigatoriamente terminar com \textit{endfor}.

Deste modo, apresenta-se como:
\begin{minted}{bash}
    $for(cond)$
        for_body $SEP$ separador
    $endfor$
\end{minted}

Adicionalmente, pode-se colocar um \textbf{separador} a ser acrescentado
em cada iteração à frente do corpo do ciclo, com a exceção da última.
Para tal foi criada a produção \textit{Sep}.
O separador é opcional, por isso tem um campo vazio.

\begin{minted}{bash}
Sep     -> SEP TEXT
        | Empty
\end{minted}

\subsubsection{\textit{StmtSubtemplate}} \label{grm:sub}
Para permitir utilizar outros \textit{templates} e integrar
o seu texto na compilação em execução, adicionou-se a produção \textit{StmtSubtemplate}.

Utilizou-se os parênteses $()$ como sinal para a utilização de um \textit{subtemplate}.

\begin{minted}{bash}
    $subtemplate()$
    $subtemplate()\pipe$
\end{minted}

Adicionalmente, estes podem sofrer transformações através de \textit{pipes} (\ref{grm:pipes}),
utilizando o não terminal, no final da produção.
\dots \textit{pipes} devem afetar toda a produção, pelo que esta deve ser 
percorrido pelo \textit{parser}, como o \textit{template} principal.

Uma distinção dos \textit{subtemplates}
teve de ser feita para permitir o uso a partir
de um iterável, $It$ (\ref{grm:it}),
nomeadamente no interior de ciclos \textit{for}, não sendo permitidos \textit{pipes}.
Assim, foi criada a produção \textit{SubIt}.

\subsubsection{\textit{It}} \label{grm:it}
\begin{minted}{bash}
It      -> IT ItOpt Pipes Newline
ItOpt   ->  COLON SubIt                                
        |   DOT VarAtomic COLON SubIt                        
        |   DOT VarAtomic                                    
        |   DOT VarAtomic OSQBRAC NUM COMMA NUM CSQBRAC      
        |   OSQBRAC NUM COMMA NUM CSQBRAC              
        |   Empty  
\end{minted}
Para permitir, dentro das operações de \textbf{metadados}, aceder ao conteúdo do objeto
a ser iterado, utiliza-se a diretiva \textit{it},
representada na produção \textit{It}.

\begin{minted}{bash}
    $it$
    $it:subtemplate()$
    $it.var:subtemplate()$
    $it.var$
    $it.var[begin_range, end_range]$
    $it.[begin_range, end_range]$
\end{minted}

Neste caso, o \textit{it} permite aceder ao objeto concreto a ser processado,
assim como aos seus constituintes.
São utilizados com grande efeito dentro dos ciclos \textit{for}, 
para permitir obter o próximo valor dentro de um 
iterável, em cada nova iteração do \textit{loop}.

% É notável a fatorização à esquerda efetuada entre todas as possíveis 
% opções do \textit{it}.

\subsubsection{\textit{Nesting}} \label{grm:nest}
\begin{minted}{bash}
Nesting ->  Var TEXT '^' NestElems '^' Newline

NestElems -> NestElems
          |  NestElem

NestElem  -> Var
          |  TEXT
\end{minted}
Para permitir conteúdo \textit{nested} dentro do documento, i.e
a preservação da identação das linhas subsequentes, 
foram criadas as produções para o \textit{nesting}.

Utiliza-se a diretiva \mintinline{bash}{'^'} para separar o elemento para o qual todas as 
linhas se vão indexar, \textit{Var}, e o restante texto, em \textit{NestElems}.

Para tomar partido do reconhecedor utilizado pelo \textit{yacc}, 
definiu-se à esquerda a recursividade da lista de elementos, \textit{NestElems}.
Estes podem ser \textbf{texto} ou \textbf{variáveis}.

\dots \textit{NestElems} representa uma parte da produção 
\textit{elems} (\ref{grm:elems}). Este foi necessário para evitar problemas 
com \textbf{reduções}, sendo provocada uma circularidade não controlável.

\subsubsection{\textit{Conds}} \label{grm:conds}
As variáveis utilizadas na entrada das operações de \textit{If} e
\textit{For} são apresentadas com a produção \textit{cond}.

No contexto das operações \textit{for} é dada a opção 
de inserir um \textit{range} de valores; para permitir
mais flexibilidade ao utilizador.

A título sugestivo, seria relevante acrescentar um conjunto de produções
que permitisse a utilização de operadores lógicos.
Tal implementação não é complexa, mas requer uma nova iteração de trabalho.

\subsubsection{\textit{Var}} \label{grm:vars}
O elemento atómico para representar variáveis no documento.

\begin{minted}{bash}
Var -> VarAtomic Pipes

VarAtomic ->  VarAtomic DOT ID                          
          |  ID   
\end{minted}

Uma variável é caracterizada pelo seu nome, um \textit{ID}.
Contudo, no caso de dicionários, os valores devem ser acedidos 
com $ID.key\_name$.
Neste sentido, está-se a manipular os valores cujo nome da 
chave no mapa é \textit{key\_name}.

Mais tarde, devido à implementação de \textit{pipes} (\ref{grm:pipes}), 
as produções foram alteradas para permitir uma variável 
ser modificada; foi criada \textit{VarAtomic} para representar
apenas as variáveis.
 
\subsubsection{\textit{Pipes}} \label{grm:pipes}
\begin{minted}{bash}
Pipes   -> Pipes SLASH ID                                
        | Pipes SLASH ID NUM QUO TEXT QUO                
        | Pipes SLASH ID NUM QUO TEXT QUO QUO TEXT QUO   
        | Empty   
\end{minted}
Foram implementados um conjunto de \textit{pipes}, 
como apresentados em \ref{subsec:pipes}.

É permitido uma \textbf{sequência de pipes}, sendo a produção
\textit{pipes} uma lista.
Por esse motivo, existe um caso \textbf{vazio}.

Neste caso, podem ser apresentados como:
\begin{minted}{bash}
    \uppercase
    \uppercase\center 20 "|" "|"
\end{minted}
\dots adicionalmente apresenta-se \textit{pipes} com argumentos, sendo 
dinâmicos no seu \textit{output}.

A adição dos \textit{pipes} nas operações superiores provocou 
alguns problemas do âmbito \textit{reduce-reduce} e \textit{shift-reduce} devido à circularidade
causada pela reutilização de produções.
Por esse motivo, a sua possibilidade de utilização nas várias produções teve de ser 
devidamente revista e ajustada, para garantir o bom funcionamento do sistema.

\subsubsection{\textit{Range} e \textit{Num}} \label{grm:num}
\begin{minted}{bash}
Num     -> NUM
        | Empty

Range   -> OSQBRAC RangeOpt CSQBRAC

RangeOpt -> NUM COLON NUM
         |  NUM COLON NUM COLON NUM
\end{minted}
Dado a ser possibilitada a utilização de \textit{range} nos iteráveis ($It$),
foi criada uma produção para representar os parâmetros recebidos, do tipo numérico.

A necessidade advém de ser oferecida a opção de não inserir um valor, para representar
os extremos de um objeto a iterar.
Por esse motivo, \textit{Num} têm uma produção a vazio.

\subsubsection{\textit{Newline}} \label{grm:nl}
A produção é utilizada para o tratamento dos carateres \textit{newline}
em todo o ficheiro.
Ao nível da análise léxica são contabilizados os carateres \textit{newline}
lidos no ficheiro original, sendo armazenado o total.
Posteriormente, serão adicionados no texto final, nas 
posições apropriadas para re-estabelecer o formato do \textit{template}.

É possível não existir nenhum carater \textit{newline} no ficheiro original,
no contexto de uma produção da gramática, pelo que existe o caso \textbf{vazio}.

A implementação desta produção nos vários locais da gramática, foi 
bastante delicada, tendo causado um conjunto de problemas do tipo 
\textit{shift-reduce}.
Por outro lado, atendendo que há situações em que se pretende ignorar a nova linha
e outras em que se pretende ter conhecimento da sua existência, acrescentou-se à classe
\textbf{elem} uma variável \textbf{end} que simboliza o carater de fim desse elemento 
e permite modificar o comportamento do $print$.

\end{document}