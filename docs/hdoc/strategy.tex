\documentclass[../relatorio.tex]{subfiles}
\begin{document}
% - estratégia utilizada
% -- arvore de parsing concreta
% -- classes
% -- hierarquia
% -- descentralização das responsabilidades
% -- diagrama de classes

No âmbito de aplicar as regras da gramática definidas anteriormente,
recorreu-se ao uso do módulo \textit{Yacc} da biblioteca \textit{PLY}. 
Deste modo, construiu-se um \textit{parser}, estabelecendo a análise 
sintática a efetuar.

\subsection{Árvore de \textit{Parsing} Concreta}

%árvore de parsing concreta
Apesar de ser sugerido pela equipa docente uma utilização de compilação 
de código \textit{python} intermédio, o grupo preferiu
abraçar o desafio de construir uma árvore de parsing.
Esta decisão surgiu do facto das mesmas serem bastante flexíveis,
permitindo, com pequenos ajustes, transformar a ação semântica que se está 
a utilizar.

Assim, por forma a tornar o código mais fácil de desenvolver, interpretar e manter,
utilizou-se o paradigma de programação orientada aos objetos e a sua construção
hierárquica.
A definição das classes necessárias, derivou da gramática abstrata construída inicialmente.


\subsection{Classes}

O diagrama de classes, na sua totalidade, encontra-se definido em \ref{sec:classes}.

\subsubsection{\textit{Var}} \label{subsec:var}
\subfile{../hdoc/var.tex}

\subsubsection{\textit{If}} \label{subsec:If}
\subfile{../hdoc/ifs.tex}

\subsubsection{\textit{For}}\label{subsec:For}
\subfile{../hdoc/for.tex}

\subsubsection{\textit{Subtemplates}} \label{subsec:subt}
\subfile{../hdoc/subtempl.tex}

\subsubsection{\textit{Pipes}} \label{subsec:pipes}
\subfile{../hdoc/pipes.tex}

\subsubsection{\textit{It}}\label{subsec:it}
\subfile{../hdoc/it.tex}

\subsubsection{Nesting}
\subfile{../hdoc/nest.tex} \label{subsec:nest}

\subsubsection{Entity}
\subfile{../hdoc/entity.tex} \label{subsec:entity}

\subsubsection{\textit{Comments}} \label{subsec:comments}
\subfile{../hdoc/comments.tex}

\end{document}