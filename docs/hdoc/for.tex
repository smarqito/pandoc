\documentclass[../relatorio.tex]{subfiles}
\begin{document}
    Um ciclo \textit{for} é iniciado pela palavra reservada \textit{for}
    seguido de uma variável entre parênteses(dentro de '\$'), terminando 
    com um \textit{endfor}(dentro de '\$').
    Caso a variável a iterar seja:
    \begin{itemize}
        \item \textbf{Lista: }Os elementos são percorridos ordenadamente,
        sendo a variável definida para cada elemento.
        \item \textbf{Dicionário: }As chaves são percorridas ordenadamente,
        sendo a variável definida para o valor correspondente. 
        \item \textbf{Texto: }O ciclo só irá realizar uma única iteração.
    \end{itemize}

    \begin{minted}{bash}
        $for(example)$
            $example.ex1$, $example.ex2$
        $endfor$

        $for(list)$$list$$endfor$
    \end{minted}

    Caso se pretenda utilizar um separador entre valores consecutivos numa
    iteração do ciclo \textit{for}, é possível utilizar o sep(dentro de '\$'),
    indicando o separador a utilizar. Por exemplo:
    Caso queiramos iterar a lista:

    \mintinline{Python}{autores = ["José", "Marco", "Miguel"]}

    no seguinte template:

    \mintinline{Bash}{$for(autores)$autor = $autores$$sep$, $endfor$}

    Será produzido:

    \mintinline{Bash}{autor = José, autor = Marco, autor = Miguel}

    Adicionalmente, foi criado um \textit{Range} que pode ser utilizado como a
    condição do ciclo \textit{for}. Este é defenido entre parênteses retos e 
    contem três número dividos por virgulas. O primeiro corresponde ao minímo, o segundo
    ao máximo e o terceiro ao \textit{step}. O primeiro, segundo e terceiro argumento são
    por defeito 0, infino e 1, respetivamente.
    Por exemplo:
    \mintinline{Bash}{$for([0, 3])$Hello World $endfor$}
    produz
    \mintinline{Bash}{Hello World Hello World Hello World }

    \subsubsection{Estratégia utilizada}

    A classe \textit{StmtFor} é inicializada com a variável a iterar, a lista
    de elementos que compõe o corpo do \textit{for} e o separador no caso de
    este ser utilizado.

    Uma vez que a forma de iterar o ciclo, depende do tipo da variável, tal como
    já foi explicado anteriormente, foram criadas 3 funções: \textit{handleStr},
    \textit{handleList} e \textit{handleDict}. 

    Em todas as funções são percorridos todos os elementos da variável, sendo estes imprimidos
    consoante o valor a ser iterado. 

    
    No caso das listas(\textit{handleList}) é passado à função que imprime os objetos \textit{Elem} o elemento e também a condição.
    Assim, caso no corpo apareça a variável da condição pertende-se imprimir
    apenas o valor e não a lista inteira. No caso de existir um separador, este é
    impresso em todas as iterações, excepto na última. 

    A função \textit{handleDicts} é análoga à função responsável pelas listas, divergindo
    apenas no valor iterado, que no caso dos dicionário são apenas os valores e também nos
    argumentos passados à função que imprime os elementos. No caso dos dicionários, caso se
    pretenda imprimir o valor a ser iterado é utilizada a palavra reservada \textit{it}, que
    será abordada em \ref{subsec:sep}. 

    A \textit{handleStr} é mais simples, faz apenas uma iteração passando o valor da variável 
    à função que imprime os elementos que constituem o seu corpo.


\end{document}
