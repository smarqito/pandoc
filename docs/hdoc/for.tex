\documentclass[../relatorio.tex]{subfiles}
\begin{document}
    Um ciclo \textit{for} é iniciado pela palavra reservada \textit{for}
    seguido de uma variável entre parênteses(dentro de '\$'), terminando 
    com um \textit{endfor}(dentro de '\$').
    Caso a variável a iterar seja:
    \begin{itemize}
        \item \textbf{Lista: }Os elementos são percorridos ordenadamente,
        sendo a variável definida para cada elemento.
        \item \textbf{Dicionário: }As chaves são percorridas ordenadamente,
        sendo a variável definida para o valor correspondente\dots
        \item \textbf{Texto: }O ciclo só irá realizar uma única iteração.
    \end{itemize}

    \begin{minted}{bash}
        $for(example)$
            $example.ex1$, $example.ex2$
        $endfor$

        $for(list)$$list$$endfor$
    \end{minted}

    Caso se pretenda utilizar um separador entre valores consecutivos numa
    iteração do ciclo \textit{for}, é possível utilizar o sep(dentro de '\$'),
    indicando o separador a utilizar. Por exemplo:
    Caso queiramos iterar a lista:

    \mintinline{Python}{autores = ["José", "Marco", "Miguel"]}

    no seguinte template:

    \mintinline{Bash}{$for(autores)$autor = $autores$$sep$, $endfor$}

    Será produzido:

    \mintinline{Bash}{autor = José, autor = Marco, autor = Miguel}

    \subsubsection{Estratégia utilizada}

    A classe \textit{for} é inicilizada com a variável a iterar, a lista
    de elemento que compõe o corpo do \textit{for} e o separador no caso de
    este ser utilizado.

    Uma vez que a forma de iterar o ciclo, depende do tipo da variável, tal como
    já foi explicado anteriormente, foram criadas 3 funções: \textit{handleStr},
    \textit{handleList} e \textit{handleDict}\dots

    Em todas as funções é percorrido a lista de elementos, sendo estes imprimidos
    consoante o valor a ser iterado. No caso das listas é passado o elemento e também
    a condição, desta forma, caso no corpo apareça a variavel da condição pertende-se 
    imprimir apenas o valor e não a lista inteira. 
    
    FALTA FALAR DOS MAPAS


\end{document}
