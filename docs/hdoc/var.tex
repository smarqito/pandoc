\documentclass[../relatorio.tex]{subfiles}
\begin{document}
    O elemento primitivo da camada de \textbf{metadados},
    a variável, delimitada por '\$', é utilizada para 
    aceder a qualquer objeto dentro do dicionário,
    desde que o acesso seja feito a partir do próprio nível em que a 
    iteração se encontra.

    Deste modo, considerando o dicionário:
    
    \mintinline{python}{'map': {'key1': {'key11': 'value1'}, 'key2': {'key22': 'value22'}}}

    E o respetivo comando dentro do \textit{template}:
    \begin{minted}{bash}
        Exemplo 1:
        $map$

        Exemplo 2:
        $map.key1$

        Exemplo 3:
        $map.key1.key11$

        Exemplo 4:
        $map.key2.key22$
    \end{minted}

    O texto produzido será: 

    \begin{minted}{Python}
        Exemplo 1:
        {'key1': {'key11': 'value1'}, 'key2': {'key22': 'value22'}}

        Exemplo 2:
        {'key11': 'value1'}

        Exemplo 3:
        'value1'

        Exemplo 4:
        'value2'
    \end{minted}

    \dots sendo possível aceder o valor dentro do dicionário
    independentemente do seu grau de \textbf{aninhamento}.

    \subsection*{Estratégia Utilizada}
    A classe \textit{Var} é sempre inicializada com os atributos \textit{keyword} e
    \textit{dict}.
    A \textit{keyword} corresponde ao objeto a procurar dentro do dicionário \textit{dict}.
    Dentro da classe, o nome do atributo que guarda a \textit{keyword} é \textit{id}.
    Na fase de inicialização, é colocado na variável de instância \textit{value} o valor correspondente 
    à \textit{keyword}, dentro do dicionário; caso não exista, é colocado como \textbf{None}.

    Contudo, como referido em cima, deve ser possibilitado o aninhamento de objetos para poder 
    aceder a qualquer nível do dicionário.
    Neste sentido, o atributo \textit{ids}, uma lista, armazena todas 
    as chaves de acesso ao valor pretendido, sendo inicializada como uma lista vazia.
    \dots dentro do atributo \textit{id}, vai-se construir a \textit{string} que designa a variável,
    como apresentada no \textit{template}.

    No seguimento, criou-se o método \mintinline{python}{nextValue(self, keyword)}, que 
    efetua o método $get$ no \textit{value} e atribui-lhe, também, o valor resultante.
    Por fim, atualiza os atributos \textit{id} e \textit{ids}.

    Criou-se os respetivos \textit{getters} e \textit{setters} para aceder ao valor, às 
    \textit{keywords} da variável e o seu tipo, para preservar algum encapsulamento das classes,
    assim como o paradigma adotado.

    Não obstante, os métodos específicos à classe \textit{Var}, apresenta-se outros 2,
    de \textit{pretty printing}, que a classe faz \textit{override} da \textit{super class} Elem,
    \textit{pp} e \textit{pp\_list}.
    
    Assim, dependendo do tipo da variável, um dos métodos será utilizado, contendo uma 
    estratégia diferente; pelo que em ambos os casos têm a respetiva verificação de erros.

    Como consequência da adição dos \textit{pipes} (\ref{subsec:pipes}), ainda existe um último 
    atributo, \textit{pipes}, inicializado a \textit{None}.

\end{document}