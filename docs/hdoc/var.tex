\documentclass[../relatorio.tex]{subfiles}
\begin{document}
    O elemento primitivo da camada de \textbf{metadados},
    a variável, delimitada por '\$', é utilizada para 
    acessar um qualquer objeto dentro do dicionário.

    Deste modo, considerando o dicionário:
    
    \mintinline{Python}{'map': {'key1': {'key11': 'value1'}, 'key2': {'key22': 'value22'}}}

    E o respetivo comando dentro do \textit{template}:
    \begin{minted}{bash}
        Exemplo 1:
        $map$

        Exemplo 2:
        $map.key1$

        Exemplo 3:
        $map.key1.key11$

        Exemplo 4:
        $map.key2.key22$
    \end{minted}

    O texto produzido será do tipo: 

    \begin{minted}{Python}
        Exemplo 1:
        {'key1': {'key11': 'value1'}, 'key2': {'key22': 'value22'}}

        Exemplo 2:
        {'key11': 'value1'}

        Exemplo 3:
        'value1'

        Exemplo 4:
        'value2'
    \end{minted}

    \dots sendo possível acessar o valor dentro do dicionário
    independentemente do seu grau de \textbf{aninhamento}.

    \subsection{Estratégia Utilizada}
    A classe \textit{Var} é sempre inicializada com os atributos \textit{keyword} e
    \textit{dict}.
    A \textit{keyword} corresponde ao objeto a procurar dentro do dicionário \textit{dict}.
    Dentro da classe, o nome do atributo que guarda a \textit{keyword} é \textit{id}.
    Na fase de inicialização, é colocado em \textit{value} o valor correspondente 
    à \textit{keyword}, dentro do dicionário; caso não exista é colocado como \textbf{None}.

    Contudo, como referido em cima, deve ser possibilitado o aninhamento de objetos para poder 
    acessar um valor do dicionário.
    Neste sentido, dentro do atributo \textit{ids}, uma lista, irá encontrar-se todas 
    as \textit{keywords} a serem adicionadas à variável, sendo inicializada com a primeira
    \textit{keyword} passada.
    \dots dentro do atributo \textit{id} vai construir-se a \textit{string} da variável, como 
    apresentada no \textit{template}.

    No seguimento criou-se o método \mintinline{Python}{nextValue(self, keyword)}, que 
    coloca em \textit{value} o objeto resultante, acessado por \textit{dict}, e 
    atualiza os atributos \textit{id} e \textit{ids}.

    Criou-se os respetivos \textit{getters} e \textit{setters} para acessar o valor, as 
    \textit{keywords} da variável e o seu tipo, para preservar a estratégia de um \textbf{paradigma
    orientado a objetos} adotada.

    Não obstante os métodos específicos à class \textit{Var}, apresenta-se outros 2, de \textit{pretty printing},
    que a classe faz \textit{override} da \textit{super class} Elem, \textit{pp} e \textit{pp\_list}.
    
    Assim, dependendo do tipo da variável, um dos métodos irá ser utilizado, contendo uma 
    apresentação diferente; pelo que em ambos os casos têm a respetiva verificação de erros.

    Como consequência da adição dos \textit{pipes} (\ref{subsec:pipes}), ainda existe um último 
    atributo, \textit{pipes}, inicializado a \textit{None}.

\end{document}