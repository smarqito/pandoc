\documentclass[../relatorio.tex]{subfiles}
\begin{document}
    Tal como no \textit{Pandoc}, foram criados \textit{Pipes}
    que transformam o valor de uma variável ou de um \textit{Partial}.

    Os \textit{Pipes} são especificados com o uso de uma barra (/), 
    depois do nome da variável ou do \textit{partial}. Por exemplo:

    \begin{minted}{bash}
        $author/uppercase$

        $for(authors/reverse)$
            $it.name$
        $endfor$

        $subtemplate()/lowercase$
    \end{minted}

    Os \textit{Pipes} também podem ser encadeados:
    \begin{minted}{bash}
        $author/uppercase/reverse$
    \end{minted}
    
    Quanto aos \textit{Pipes} criados, foram feitos todos os que o
    \textit{Pandoc} tem implementados até ao momento da realização
    do presente relatório, mais concretamente:

    \begin{itemize}
        \item \textit{pairs: }Converte um dicionário ou uma lista
        para um dicionário de listas. Caso o valor seja uma lista,
        a chave corresponde ao índice do array, começando em 1.
        \item \textit{uppercase: }Converte um texto todo para
        letras maiúsculas.
        \item \textit{lowercase: }Converte um texto todo para
        letras minúsculas.
        \item \textit{length: }Retorna o tamanho do valor, no 
        caso de um valor textual o número de caracteres, numa
        lista ou dicionário o número de elementos.
        \item \textit{reverse: }Inverte um valor textual ou uma
        lista
    \end{itemize}

\end{document}