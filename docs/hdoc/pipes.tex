\documentclass[../relatorio.tex]{subfiles}
\begin{document}
    Tal como no \textit{Pandoc}, foram criados \textit{Pipes}
    que transformam o valor de uma variável ou de um \textit{Partial}.

    Os \textit{Pipes} são especificados com o uso de uma barra (/), 
    depois do nome da variável ou do \textit{partial}. Por exemplo:

    \begin{minted}{bash}
        $author/uppercase$

        $for(authors/reverse)$
            $it.name$
        $endfor$

        $subtemplate()/lowercase$
    \end{minted}

    Os \textit{Pipes} também podem ser encadeados:
    \begin{minted}{bash}
        $author/uppercase/reverse$
    \end{minted}
    
    Quanto aos \textit{Pipes} criados, foram feitos todos os que o
    \textit{Pandoc} tem implementados até ao momento da realização
    do presente relatório, mais concretamente:

    \begin{itemize}
        \item \textit{pairs: }Converte um dicionário ou uma lista
        para um dicionário de listas. Caso o valor seja uma lista,
        a chave corresponde ao índice do array, começando em 1.
        \item \textit{uppercase: }Converte um texto todo para
        letras maiúsculas.
        \item \textit{lowercase: }Converte um texto todo para
        letras minúsculas.
        \item \textit{length: }Retorna o tamanho do valor, no 
        caso de um valor textual o número de caracteres, numa
        lista ou dicionário o número de elementos.
        \item \textit{reverse: }Inverte um valor textual ou uma
        lista.
        \item \textit{first: }Retorna o primeiro elemento de uma
        lista.
        \item \textit{last: }Retorna o último elemento de uma
        lista.
        \item \textit{rest: }Retorna o todos os elementos de uma
        lista menos o primeiro.
        \item \textit{allbutlast: }Retorna o todos os elementos de uma
        lista menos o último.
        \item \textit{chomp: }Substitui \textit{newlines} seguidos por apenas
        um. 
        \item \textit{nowrap: }
        \item \textit{alpha: }Em valores textuais, converte números inteiros, 
        entre 1 e 26, para a correspondente letra minúscula. 
        \item \textit{roman: }Em valores textuais, converte número inteiros para
        números romanos minuscúlos. 
        \item \textit{left n "leftboarder" "rigthboarder": }Coloca um valor textual
        num bloco de tamanho n, alinhado à esquerda, com um limite opcional à esquerda
        e à direita. Caso só seja passado um limite, por defeito assume-se que se trata
        do limite à esquerda. 
        \item \textit{rigth n "leftboarder" "rigthboarder": }Coloca um valor textual
        num bloco de tamanho n, alinhado ao centro, com um limite opcional à esquerda
        e à direita. Caso só seja passado um limite, por defeito assume-se que se trata
        do limite à esquerda. 
        \item \textit{center n "leftboarder" "rigthboarder": }Coloca um valor textual
        num bloco de tamanho n, alinhado à direita, com um limite opcional à esquerda
        e à direita. Caso só seja passado um limite, por defeito assume-se que se trata
        do limite à direita. 
    \end{itemize}

    Tendo como variáveis:
    \begin{minted}{Python}
        autores : {ent1 : {nome : Jose, numero : 93271, turno: PL2, nota: 20}, 
                   ent2 : {nome : Marco, numero : 62608, turno: PL3, nota: 20},
                   ent3 : {nome : Miguel, numero : 94269, turno: PL5, nota: 20}}
    \end{minted}
    E o seguinte \textit{template}:
    \begin{minted}{bash}
        |$for([0:9])$-$endfor$|$for([0:9])$-$endfor$|$for([0:9])$-$endfor$|$for([0:9])$-$endfor$|
        |Nome     |Nº       | Turno   |     Nota|
        |$for([0:9])$-$endfor$|$for([0:9])$-$endfor$|$for([0:9])$-$endfor$|$for([0:9])$-$endfor$|
        $for(obj.autores)$
        $it.nome/uppercase/left 9 "|"$$it.numero/left 9 "|"$$it.turno/roman/uppercase/center 9 "|" "|"$$it.nota/right 9 "|"$
        |$for([0:9])$-$endfor$|$for([0:9])$-$endfor$|$for([0:9])$-$endfor$|$for([0:9])$-$endfor$|
        $endfor$
    \end{minted}
    O texto gerado é:
    \begin{minted}{bash}
        |---------|---------|---------|---------|
        |Nome     |Nº       | Turno   |     Nota|
        |---------|---------|---------|---------|
        |JOSE     |93271    |   PLII  |       20|
        |---------|---------|---------|---------|
        |MARCO    |62608    |  PLIII  |       20|
        |---------|---------|---------|---------|
        |MIGUEL   |94269    |   PLV   |       20|
        |---------|---------|---------|---------|
    \end{minted}

    \subsubsection{Estratégia utilizada}

    A classe \textit{Pipes} é inicializada com uma lista de objetos da classe 
    \textit{Pipe}. Tem ainda uma função, \textit{handlePipes}, que aplica iterativamente
    cada \textit{pipe} ao valor ou \textit{partial} recebido, retornando o valor transformado.

    Na classe \textit{Pipe} foi criado um dicionário, em que a chave é o nome do \textit{pipe} 
    e o valor é a função que implementa o respetivo método. 
    
    \inputminted[firstline=214, lastline=231]{py}{../modules/Pipe.py}
    
    Desta forma, a função responsável por aplicar o \textit{pipe} ao valor apenas precisa
    de fazer:
    \begin{minted}{Python}
        Pipe._pipes[self.id](args)
    \end{minted}
    \dots sendo o self.id o nome do \textit{pipe} e args os argumentos passados, podendo variar de
    $1$ a $4$. 

    A classe \textit{Elem} tem as funções responsáveis por guardar os pipes associados às variáveis
    ou aos \textit{partials}, assim na parte da escrita do texto final no caso de estes existirem,
    estes são alterados para os novos valores. Consequentemente, todas as subclasses de \textit{Elem} tem
    acesso a estas funções.

    A função \textit{set\_pipes} é chamada no \textit{parser} de maneira a guardar os \textit{pipes} lidos. 
    Posteriormente, na impressão do texto estes pipes são aplicados com a função \textit{apply\_pipes} 
    que recebe o valor original e
    retorna o valor transformado.
    %Falta falar do add_pipe

\end{document}