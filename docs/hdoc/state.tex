\documentclass[../relatorio.tex]{subfiles}
\begin{document}
    No âmbito de efetuar o \textit{tokenize} do ficheiro \textit{input},
    para posteriormente ser tratado ao nível do \textit{parser}, 
    utilizou-se o módulo \textit{Lex} da biblioteca \textit{PLY}.
    Deste modo, será estabelecida a rotina de análise léxica 
    dos \textit{templates} a utilizar, verificando anomalias e
    proporcionando os parâmetros corretos para a construção da árvore 
    sintática mais tarde.

    \subsubsection{Símbolos utilizados}

    Seguem-se os seguintes \textit{tokens}:
    \begin{itemize}
        \item \textit{ID :} Um conjunto de caracteres, sendo permitido no ínicio o "\_".
        \item \textit{TEXT :} Qualquer conjunto de caracteres excluindo - "\$" ,"$\backslash$" e "$\backslash$n".
        \item \textit{OPAR CPAR :} Abertura e fecho de parênteses, respetivamente.
        \item \textit{BACK :} O caracter "$\backslash$".
        \item \textit{COLON :} O caracter ":".
        \item \textit{OSQBRAC CSQBRAC :} Abertura e fecho de parênteses retos.
        \item \textit{COMMA :} O caracter ",".
        \item \textit{DOT :} O caracter ".".
        \item \textit{SLASH :} O caracter "/".
        \item \textit{NUM :} Qualquer conjunto de números.
        \item \textit{QUO :} O caracter " (aspas).
        \item \textit{NL :} Conjunto de \textit{newlines} - "$\backslash$n".
        \item \textit{OCOMMENT CCOMMENT: } Abertura e fecho de um comentário.
        \item \textit{COMMENT: } Qualquer conjunto de caracter até encontrar "-->"
    \end{itemize}

    Adicionadas aos \textit{tokens}, mas cujo o significado foi abordado
    no contexto da gramática (\ref{subsec:grammar}), as palavras reservadas:
        \textbf{if else elseif endif for endfor sep it}

    Por fim, recorreu-se ao uso de um \textbf{literal}, com "\^{}".

    \subsubsection{Estados Criados}
    No âmbito da definição da linguagem usada nos \textit{templates},
    foi necessária a implementação de \textbf{três estados para contexto} no
    analisador léxico.

    \inputminted[firstline=13, lastline=17]{python}{../pandoc_lex.py}

    O primeiro, \textbf{metadata}, é utilizado para identificar todos os
    campos de metadados, eg. \textbf{operações If/For}.
    O caracter que simboliza a entrada, e a consequente saída, é o 
    \textit{dollar sign} ("\$").
    O estado pode ser acedido quando se encontra na contexto inicial, e
    na saída do estado \textit{args}.

    O segundo, \textbf{args}, foi adicionado mediante a apresentação 
    de argumentos nos \textit{pipes} (\ref{subsec:pipes}), para permitir 
    distinguí-los do texto padrão. 
    Neste caso, entra-se no contexo \textit{args} apenas dentro do 
    estado \textit{metadata}, ao aparecer o símbolo \textbf{"} (aspas).
    Ao processar novamente o símbolo \textit{aspas} volta ao estado \textit{metadata},
    como referido anteriormente.

    O terceiro, \textbf{comments}, foi utilizado mediante a possibilidade de 
    encontrar comentários nos templates. Dessa maneira, foi necessário criar um 
    estado para compreender e diferenciar estes do texto regular.

    \dots os três estados são exclusivos, pois é necessário fazer \textit{override}
    do comportamento padrão do \textit{lexer}.
    Nestes, apenas são válidos um conjunto de caracteres, daí a sua exclusividade e 
    necessidade de definir todas as regras válidas em cada.

    No seguimento da definição dos estados, surge um diagrama de 
    \textbf{máquina de estados} para permitir uma melhor compreensão da
    dinâmica léxica, mediante os estados existentes, em \ref{fig:state_machine}.

    \begin{figure}[!ht]
        \centering
        \includegraphics[width=\linewidth]{assets/Máquina_estados.png}
        \caption{Diagrama de máquina de estados para o analisador léxico.}
        \label{fig:state_machine}
    \end{figure}

\end{document}