\documentclass[../relatorio.tex]{subfiles}
\begin{document}
    O \textit{Pandoc} permite que um \textit{template} faça referência a outro
    através de um \textit{Partial}, isto é, um \textit{Subtemplate}. A utilização
    de um \textit{Subtemplate} é feita através do nome do ficheiro seguido de '()'
    (dentro de '\$'). Por exemplo:

    \mintinline{bash}{$subtemplate()$}

    O ficheiro chamado, tem de estar obrigatoriamente na mesma diretoria do \textit{template}
    principal. Caso não seja passado a terminação do ficheiro, por defeito assume-se que o
    \textit{subtemplate} tem a mesma terminação que o principal, isto é, para o exemplo acima, caso o nome do
    \textit{main template} fosse 'template.html' assumia-se que o \textit{subtemplate} tinha o nome "subtemplate.html".
    
    Os \textit{Partials} podem ser aplicados apenas a uma variável do dicionário, bem como podem ser seguidos de pipes:

    \begin{minted}{bash}
        $for(obj)$
        $it:subtemplate2()/uppercase$
        $endfor$
    \end{minted}

    Os \textit{substemplates} podem também conter outros \textit{subtemplates}.

    \subsection{Estratégia utilizada}

    Para iterar um \textit{subtemplate} é preciso criar um novo \textit{Parser} a partir do
    comando \mintinline{Python}{yacc.yacc()}. \textit{Parser} este que é passado como argumento
    à classe \textit{StmtSubtemplate}, bem como o nome do ficheiro e opcionalmente uma lista
    de \textit{KeyWords} e uma lista de \textit{Pipes}. 
    
    O \textit{Parser} passado contém um argumento denominado \textit{finfo}, que se trata um dicionário com a informação
    sobre o \textit{path} e também a terminação do \textit{template} principal e ainda o nome do \textit{subtemplate}. 

    Quando a classe é inicializada é chamada a função \textit{handleFileName}, que atualiza o \textit{finfo} com a 
    informação do \textit{subtemplate}.

    \inputminted[firstline=14, lastline=21]{py}{../modules/Stmt/StmtSubtemplate.py}

    Tal como referido anteriormente, caso não seja passado a terminação do \textit{subtemplate}, assume-se que 
    é a mesma do ficheiro que o chamou. Para obter o nome do ficheiro, foi utilizada uma expressão regular que captura
    todas os caracteres, excepto os que não são permitidos em nomes de ficheiros.

    Sabendo que o dicionário \textit{finfo} é constítuido pelas chaves 'path', 'fname' e 'ext', para obter o \textit{path}
    completo para o \textit{subtemplate} é só concatenar os valores de cada, isto é:

    \mintinline{Python}{path = finfo['path'] + finfo['fname'] + finfo['ext']}

    A partir deste \textit{path} completo e com o novo parser passado por argumento, é possível ler o conteúdo do ficheiro
    e fazer o \textit{parsing} do mesmo. 
    
    Importante ainda referir que caso a classe tenha sido inicializada com uma lista de
    \textit{Pipes} é utilizada a função \textit{addPipes} para passar estes pipes ao novo objeto \textit{Document} criado,
    desta forma, quando é efetuada a escrita do ficheiro, previamente já foram aplicados os \textit{Pipes}.
\end{document}