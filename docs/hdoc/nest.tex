\documentclass[../relatorio.tex]{subfiles}
\begin{document}
    É utilizado o símbolo \textbf{literal} "\^{}" para 
    colocar conteúdo \textit{nested}, i.e. obrigar que
    as subsequentes linhas estejam alinhadas com a variável 
    escolhida; preserva a identação da primeira variável.

    Com base no dicionário:

    \begin{minted}{Python}
        'map':{'var1' : "Lorem ipsum" , 
               'var2' : "sit amet, consectetur adipiscing elit, sed do eiusmod
                         tempor incididunt ut labore et dolore magna aliqua. 
                         Id neque aliquam vestibulum morbi blandit cursus."}
    \end{minted}

    \dots e o exemplo num \textit{template}:

    \begin{minted}{bash}
        $var1$$^$ $var2$ $^$
    \end{minted}
                           
    O texto obtido segue-se:
    \begin{minted}{bash}
        Lorem ipsum  sit amet, consectetur adipiscing elit, sed do eiusmod
                     tempor incididunt ut labore et dolore magna aliqua. 
                     Id neque aliquam vestibulum morbi blandit cursus.
    \end{minted}

    \dots pelo que o texto da variável \textit{var2} vai sempre começar 
    à frente de \textit{var1}; encontra-se \textit{indented} pelo prefixo 
    definido pela \textit{var1}.
    
    \subsection*{Estratégia Utilizada}
    
    A classe \textit{Nesting} contém no seu atributo \textit{preffix} a 
    variável cujos espaços vão ser utilizados para indexar as subsequentes
    linhas na sua correta posição.
    Em \textit{elems} encontra-se todo o texto, incluindo \textit{variáveis},
    que seguem o prefixo.
    Por fim, guarda-se no atributo \textit{spaces} o número de espaços a adicionar
    no começo de cada nova linha; corresponde ao tamanho do prefixo passado,
    sendo adicionados dois espaços para melhoria estética.
    
    Não obstante a inicialização da classe, só é feito um \textit{override}
    do método de \textit{pretty printing} (\textit{pp}), da classe 
    \textit{super} - \textit{Elem}, para apresentar corretamente 
    o conteúdo \textit{nested}.
    
\end{document}