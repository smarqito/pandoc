\documentclass[../relatorio.tex]{subfiles}
\begin{document}
    Pode-se escrever um comentário, utilizando as etiquetas \mintinline{bash}{'$--'} e \mintinline{bash}{'--$'}.
    Todo o texto que estiver no interior, será interpretado como sendo um comentário,
    não lhe sendo efetuado qualquer tratamento.
    Por defeito os comentários são ignorados quando se gera o ficheiro de \textit{output}.
    No entanto, esse comportamento pode ser alterado pelo utilizador.

    \begin{minted}{bash}
        $-- Isto é um comentário
        com 2 linhas --$

        Texto $--outro comentário--$
    \end{minted}

    Para além do utilizador ter a possibilidade de escolher entre ignorar, ou não,
    os comentários, pode também controlar a forma como este é apresentado no ficheiro gerado.
    Em particular, pode definir um prefixo e/ ou um sufixo que irão encapsular o comentário.

    Por exemplo, caso o modo de abertura e de 
    encerramento fossem, respetivamente, '\mintinline{html}{<!--}' e '\mintinline{html}{-->}', o texto gerado seria:

    \begin{minted}{bash}
        <!-- Isto é um comentário
        com 2 linhas -->

        Texto <!--outro comentário-->
    \end{minted}

    Este comportamento permite, por exemplo, gerar um ficheiro \textit{html}
    em que se mantém um conjunto de comentários no ficheiro gerado, sem alterar o 
    comportamento do \textit{layout} pretendido.

    \subsection*{Estratégia utilizada}

    Quando o utilizador define que pretende ver os comentários no ficheiro gerado,
    através da utilização do argumento \mintinline{bash}{-c}, é injetado
    no parser uma variável a \mintinline{python}{True} que indica essa pretensão.
    Adiciona-se, ainda, o prefixo e o sufixo a ser utilizado durante a geração do texto
    final.

    Assim, a classe \textit{Comment} é inicilizada com o conteúdo do comentário,
    a flag que indica se é para escrever o comentário e o 
    o modo de abertura e de encerramento do comentário (podendo este ser alterado pelo
    utilizador através das flags \mintinline{bash}{--oc='<!--'} e \mintinline{bash}{--cc='-->'}).

    A função responsável por imprimir o texto gerado, apenas escreve o
    comentário no caso da flag ser \textit{True}, concatenando com o prefixo
    e sufixo recebido.

\end{document}