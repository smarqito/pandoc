\documentclass[../relatorio.tex]{subfiles}
    Um comentário é tudo que está entre '<--' e '-->'. Por defeito
    os comentários não são escritos no ficheiro gerado, a menos que o
    utilizador o pretenda.

    \begin{minted}{bash}
        <-- Isto é um comentário
        com 2 linhas-->

        Texto <--outro comentário-->
    \end{minted}

    Como dito anteriormente, os comentários por defeito são excluídos
    do ficheiro gerado. Porém caso o utilizador os pretenda imprimir
    pode-o fazer, inclusive indicando opcionalmente qual os caracteres
    utilizados para abrir e terminar. 
    
    Isto é, utilizando o exemplo acima, caso o modo de abertura e de 
    encerrar fossem, respetivamente, '<!-' e '->', o texto gerado seria:

    \begin{minted}{bash}
        <!- Isto é um comentário
        com 2 linhas->

        Texto <!-outro comentário->
    \end{minted}


    \subsection*{Estratégia utilizada}

    A classe \textit{Comment} é inicilizada com o conteúdo do comentário,
    a flag que indica se é para escrever o comentário e opcionalmente o 
    o modo de abertura e de encerramento do comentário.

    A função responsável por imprimir o texto gerado, apenas escreve o
    comentário no caso da flag ser \textit{True}, concatenando com o prefixo
    e sufixo recebido.

\end{document}