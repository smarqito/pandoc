\documentclass[../relatorio.tex]{subfiles}
\begin{document}
    A condição \textit{if} começa com a palavra reservada \textit{if}
    seguida de uma variável entre parênteses(dentro de '\$'). Tem a
    possiblidade ainda de ser sucedido por um ou mais elseif, também com 
    uma variavel entre parênteses(dentro de '\$') ou então um else(dentro de '\$').
    
    O corpo da condição só é executado no caso da variável passada entre parênteses
    nao ser vazia, isto é, estar contido no YAML recebido.

    \begin{minted}{bash}
        $if(example)$
            $example$
        $endif$

        $if(example1)$
            $example1$
        $elseif(exemple2)$
            $example2$
        $else$
            Lorem ipsum    
        $endif$
    \end{minted}

    \subsubsection{Estratégia utilizada}

    Para implementar as condições \textit{if}, foram criadas 2 classes,
    uma delas a classe \textit{StmtIf} que apenas é inicializada com 
    a variável da condição e a lista de elementos que compõe o seu corpo.

    Quanto à outra classe, trata-se de uma subclasse da primeira, tendo 
    
    
\end{document}