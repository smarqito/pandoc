\documentclass[../relatorio.tex]{subfiles}
\begin{document}
    A condição \textit{if} começa com a palavra reservada \textit{if}
    seguida de uma variável entre parênteses(dentro de '\$'). Tem a
    possiblidade ainda de ser sucedido por um ou mais elseif, também com 
    uma variavel entre parênteses(dentro de '\$') ou então um else(dentro de '\$'). 

    \begin{minted}{bash}
        $if(example)$
            $example$
        $endif$

        $if(example1)$
            $example1$
        $elseif(exemple2)$
            $example2$
        $else$
            Lorem ipsum    
        $endif$
    \end{minted}

    \subsubsection{Estratégia utilizada}

    
\end{document}