\documentclass[../relatorio.tex]{subfiles}
\begin{document}
    A condição \textit{if} começa com a palavra reservada \textit{if}
    seguida de uma variável entre parênteses(dentro de '\$'). Tem a
    possiblidade ainda de ser sucedido por um ou mais elseif, também com 
    uma variavel entre parênteses(dentro de '\$') ou então um else(dentro de '\$').
    
    O corpo da condição só é executado no caso da variável passada entre parênteses
    não ser vazia, isto é, estar contido no YAML recebido.

    \begin{minted}{bash}
        $if(example)$
            $example$
        $endif$

        $if(example1)$
            $example1$
        $elseif(exemple2)$
            $example2$
        $else$
            Texto exemplificativo   
        $endif$
    \end{minted}

    \subsection*{Estratégia utilizada}

    Para implementar as condições \textit{if}, foram criadas 2 classes,
    uma delas a classe \textit{StmtIf} que apenas é inicializada com 
    a variável da condição e a lista de elementos que compõe o seu corpo.

    Quanto à outra classe, trata-se de uma subclasse da primeira, tendo 
    adicionalmente uma lista de \textit{StmtIf} que correspondem aos
    \textit{elseif} e uma lista de \textit{Elem} (\textit{elseifs}) que correspondem ao corpo
    da condição \textit{Else} (\textit{elseElems}). 

    A função responsável por imprimir um \textit{StmtIf}, em primeiro lugar testa se a sua 
    variável existe. Em caso afirmativo executa a sua lista de elementos e retorna \textit{True},
    caso contrário apenas retorna \textit{False}. 
    
    Quando se trata da classe \textit{StmtIfElse}, é inicializada uma \textit{flag} a \textit{False},
    que determina se alguma condição retornou verdadeira e em seguida é chamada a função 
    \mintinline{Python}{super().pp} que corresponde à função do \textit{StmtIf}, descrita anteriormente.

    Caso esta retorne \textit{True}, a flag é atualizada e não são testadas as restantes condições. 
    Quando esta retorna \textit{False}, são testadas as condições dos \textit{elseifs} até que uma seja verificada ou não haja mais para verificar. 
    Por último, na possiblidade de nenhuma desta se verificar e exista um else
    são executados os \textit{elseElems}.
    \begin{minted}{python}
        def pp(self):
        f = False
        f = super().pp()
        if not f:
            for elseif in self.elseifs:
                if f := elseif.pp():
                    break
        if not f:
            for elseElem in self.elseBody:
                elseElem.pp()
    \end{minted}    
\end{document}