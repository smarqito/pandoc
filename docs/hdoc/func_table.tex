\documentclass[../relatorio.tex]{subfiles}

\begin{document}
    Sucintamente, as funcionalidades proporcionadas pela 
    solução criada encontram-se apresentadas em \ref{table:funcs}.
    \begin{landscape}
        \centering
        \begin{table}[!ht]
            \begin{tabular}{|c|p{7cm}|p{6cm}|p{7cm}|}
                \hline
                Funcionalidade      
                & 
                \centering
                Descrição           
                & 
                \centering
                Sintaxe             
                & 
                Exemplo 
                \\
                \hline
                \textit{Var}        
                &
                Substitui o comando pelo valor 
                da variável no dicionário.
                Caso não exista, e não se encontre
                dentro de uma operação de validação 
                e.g. \textit{If}, ocorre um erro 
                sintático.
                &
                \$ first\_variable\_name.nested\_values... \$
                &
                \$ ucs.notas.PL \$
                \\
                \hline
                \textit{If}
                &
                Executa o \textit{body} do
                comando caso a condição seja \textit{True}.
                No caso da condição ser inválida, testa o(s)
                \textit{elseif(s)} do mesmo modo,
                senão executa o \textit{else}, caso estes 
                existam.                                \newline
                Caso contrário, a parsela é ignorada.
                &
                \$ if(cond\_name) \$                    \newline
                    if\_body                            \newline
                \$ endif \$                             \newline
                &
                \$ if(ucs.notas.PL) \$                  \newline
                    A nota óbtida é \$ucs.notas.PL\$    \newline
                    \$else\$                            \newline
                    Não existe apreciação à cadeira PL. \newline
                \$ endif \$                             \newline
                \\
                \hline
                \textit{For}
                &
                Executa o \textit{body} do 
                comando, por todas as iterações 
                da variável do ciclo.
                Se a variável não for válida, o 
                ciclo é ignorado.
                &
                \$ for(var\_name)\$                     \newline
                    for\_body                           \newline
                \$ endfor \$                            \newline
                &
                \$ for(ucs.notas) \$                    \newline
                    $it$                                \newline
                \$ endfor \$                            \newline
                \\
                \hline
                \textit{Subtemplates}
                &
                Imprime o conteúdo existente 
                num \textit{subfile}.
                Este irá ser filtrado pelo 
                \textit{parser}.
                &
                \$ subtempl\_name() \$
                &
                \$ ucs.notas() \$
                \\
                \hline
                \textit{it}
                &
                Acede ao objeto a ser iterado.
                &
                \$ it.key\_name \$
                &
                \$ for(eng\_inf) \$                    \newline
                    $it.ucs$                           \newline
                \$ endfor \$                           \newline
                \\
                \hline 
                \textit{Nesting}
                &
                Preserva a identação das linhas
                subsquentes, com base tamanho
                da primeira variável inserida.
                &
                \$ var\_name \$\$\^{}\$ \$ text\_lines \$ \$\^{}\$
                &
                \$ \$cursos.ENGINF\$ \$\$\^{}\$         \newline
                \$ \$cursos.ENGINF.alunos\$ \$ \$\^{}\$ \newline
                \\
                \hline
                \textit{Entity}
                &
                Efetua alterações no texto, 
                relativas a símbolos sensíveis e.g.
                \textit{Backslash escape} ou atalhos.
                &
                $\backslash$char\_name                     \newline
                &
                $\backslash\$$                            \newline
                $\backslash$div                           \newline
                \\
                \hline
            \end{tabular}
            \caption{Tabela de funcionalidades da aplicação}
            \label{table:funcs}
        \end{table}
    \end{landscape}
\end{document}