\documentclass[../relatorio.tex]{subfiles}
\begin{document}
    A palavra reservada \textit{it}(dentro de '\$') é utilizada
    para aceder ao objeto que está ser iterado no momento em que
    este é utilizado.

    Utilizando como exemplo o dicionário:
    
    \mintinline{python}{'map': {'key1': {'key22': 'value2'}, 'key2': {'key22': 'value22'}}}

    \begin{minted}{bash}
        $for(map)$
            $it$
        $endfor$
    \end{minted}

    O texto obtido é:
    \begin{minted}{python}
        {'key22': 'value2'}
        {'key22': 'value22'}
    \end{minted}

    A seguir da palavra reservada \textit{it}, é possível ainda indicar
    o nome da variável, bem com um \textit{Range} ou uma \textit{Subtemplate}.

    \begin{minted}{bash}
        $for(map)$
            $it.key22$
        $endfor$

        $for(map)$
            $it::subtemplate()$
        $endfor$

        $for(map)$
            $it.key22[4,]$
        $endfor$
    \end{minted}

    \subsection*{Estratégia Utilizada}

    A classe \textit{It} inicializa o seu atributo \textit{default}
    com a variavél que irá ser iterada. Para tratar dos casos em que
    aparece o nome de variável, um \textit{Range} ou uma \textit{Subtemplate}
    a seguir à palavra reservada \textit{it} foram criadas 3 classes: \textbf{ItVar}, 
    \textbf{ItRange} e \textbf{ItSubtemplate}


    A \textbf{ItVar} é uma subclasse de classe \textit{It} que recebe uma lista de \textit{Keywords}
    que correspondem ao nome das variáveis. Esta lista é iterada,
    atualizando o valor de \textit{default}. 
    
    \inputminted[firstline=11, lastline=17]{python}{../modules/It/ItVar.py}
    
    Caso o tipo do valor de \textit{default} seja um dicionário é possível obter
    a chave ou valor deste, caso as \textit{KeyWords} recebidas sejam respectivamente,
    \textit{key} e \textit{value}

    Como o \textit{Range} pode ser antecedido por nomes de variáveis, a classe
    \textit{ItRange} é uma subclasse de \textit{ItVar}. Esta recebe um \textit{Range},
    que indica o valor mínimo e máximo, atualizando o valor de default da seguinte forma:
    
    \inputminted[firstline=9, lastline=20]{python}{../modules/It/ItRange.py}

    Quanto à classe \textit{ItSubtemplate}, esta é uma subclasse da classe \textit{StmtSubtemplate}
    que será abordade em \ref{subsec:subt}. É inicializada com o nome do \textit{subtemplate},
    o novo \textit{parser} e lista de \textit{KeyWords}.

    
    
     
\end{document}