\documentclass[../relatorio.tex]{subfiles}
\begin{document}
    Para incorporar \textit{backslash escapes} dentro 
    do \textit{templates} criou-se a classe \textit{Entity}.
    Estes apresentam-se com a diretiva "$\backslash$", 
    e são utilizados para converter para \textit{string}
    símbolos sensíveis; ou então como um atalho.

    Numa fase inicial, o único exemplo para o \textit{escape}
    de um símbolo sensível é a \textit{tag} de metadados, "\$".
    O exemplo de atalho é para o ficheiro \textit{markup}, \textit{HTML},
    neste caso, uma das instruções elementares, \textit{div}.

    Como apresentado no ficheiro:
    \inputminted[firstline=8, lastline=11]{python}{../modules/Entity.py}

    A classe é inicializada com a entidade a tratar, encontrando-se no 
    atributo da classe, \textit{entity}.

    Consequentemente, utiliza-se o mapa em cima apresentado para 
    retornar a \textit{string} resultante, caso corresponda a uma das \textit{keys};
    através do método \mintinline{python}{getValue(self) -> str}.

\end{document}