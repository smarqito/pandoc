\documentclass[../relatorio.tex]{subfiles}
\begin{document}

No âmbito de um segundo trabalho, envolvendo a construção de um compilador, 
a solução apresentada consegue, com sucesso, construir novos ficheiros 
\textit{markup} através de um \textit{template} e um dicionário
\textit{YAML}.

No projeto atual o grupo utilizou a biblioteca \textit{PLY} como é o caso das aulas práticas.
Neste sentido, e com o auxílio da equipa docente, os alunos foram munidos com as ferramentas 
necessárias para poder manipular e utilizar, com sucesso, o paradigma inerente aos 
analisadores léxicos e \textit{parsers}, na construção de um compilador.

Usufruindo do paradigma orientado aos objetos, permitido na linguagem \textit{Python},
estruturou-se a construção da consequente árvore sintática de modo a ser eficiente a sua 
execução e facilitada a adição de novas funcionalidades à aplicação.

Não obstante a demonstração do domínio do panorama essencial ao segundo trabalho 
- \textit{Gramáticas} - em concreto gramáticas independentes de contexto (GIC),
o grupo também desejou evoluir as suas capacidades ao nível de programação na 
linguagem em \textit{Python}.

Deste modo, apresenta-se uma gramática sem conflitos, do qual o grupo está 
confiante, sendo colocado como a primitiva principal a eliminação 
de quaisquer problemáticas do nível \textit{shift-reduce} e, nomeadamente,
\textit{reduce-reduce}.
\end{document}